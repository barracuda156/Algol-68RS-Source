\input webmac
% This program is copyright (c) BP66 Phoenix Engineering
@Introduction.
This program started as a small utility written in OCCL \Algol. This
version uses the \Web\ system. The program converts the contents of
a file into hexadecimal.

@ cvs hex = "$Id: hex.w68,v 1.3 2005/02/17 12:45:50 sian Exp $"

@@Prelude@
BEGIN
@Included declarations@
@Names in the outer reach@
@Procedures in the outer reach@
@Logic in the outer reach@
END
@Postlude@


\M1. The local compiler requires a special prelude.

\Y\P$\4\X1:\X\S$\6
\\{PROGRAM}\\{hex}\\{CONTEXT}\\{VOID}\\{USE}\X2:Library preludes\X\par
\fi

\M2. We shall need the precompiled parameter and standard preludes.

\Y\P$\4\X2:Library preludes\X\S$\6
$\\{params},\39\\{standard}$\par
\U1.\fi

\M3. And a special postlude.

\Y\P$\4\X1:\X\mathrel{+}\S$\6
\\{FINISH}\par
\fi

\M4. The program needs the !STAT! mode and the !stat! procedure from the
\Web\ system prelude and some operators from the general prelude.

sysprelude.w
gprelude.w

\fi

\M5. The program displays the contents of the input file as follows:-
\begin{enumerate}
\item The start address of each line in hexadecimal.
\item Eight bytes printed as two hexadecimal characters for each.
\item A gap.
\item Another eight bytes as for (2).
\item On the next line, the sixteen characters as:-
\begin{enumerate}
\item 00--20 $\Rightarrow$ \verb*$\^\|X$
\item 20--7e $\Rightarrow$ \verb*\|X
\item else as \verb*.
\end{enumerate}
\end{enumerate}
All should be appropriately spaced.

Parameters.
The program accepts the following parameters:-
\begin{description}
\item[\textbf{h}] Prints a short usage message.
\item[\textbf{v}] Prints a version message.
\item[\textbf{f(filename)}] Specifies its input file.
\end{description}

Firstly, we need to open a file on !arg channel!.

\Y\P$\4\X1:\X\mathrel{+}\S$\6
$\\{open}(\\{argf},\39\.{""},\39\\{arg}\\{channel})$;\par
\fi

\M6. Here is !argf!.

\Y\P$\4\X1:\X\mathrel{+}\S$\6
\\{FILE}\\{argf};\par
\fi

\M7. And !usage!.

\Y\P$\4\X1:\X\mathrel{+}\S$\6
\4$\\{PROC}\\{usage}=\\{VOID}$: \37$(\\{put}(\\{stand}\\{error},\39(\.{"hex\
usage:\ hex\ \'(h|v|f(file))\'"},\39\\{newline}));\,\35\\{stop})$;\par
\fi

\M8. We shall be needing some error messages.

\Y\P$\4\X1:\X\mathrel{+}\S$\6
\4$\\{PROC}\\{error}=([\,]\\{CHAR}\\{msg})\\{VOID}$: \37$(\\{put}(\\{stand}%
\\{error},\39(\.{"hex:\ error:"},\39\\{msg},\39\\{newline}));\,\35%
\\{exit}(1))$;\par
\fi

\M9. We use !get param! to read and parse the parameters. No parameters
or an erroneous format simply gives the usage message.

\Y\P$\4\X1:\X\mathrel{+}\S$\6
\4$\\{CASE}\\{get}\\{param}(\\{argf})\\{IN}(\\{REF}\\{PARAM}\\{rp1})$: \37$%
\\{IF}\\{rp1}\\{IS}\\{no}\\{param}\J\\{THEN}\\{usage}\\{ELIF}\\{UPB}\\{val}%
\\{OF}\\{rp1}/=1\J\\{THEN}\\{usage}\\{ELSE}\\{CASE}(\\{val}\\{OF}\\{rp1})[1]%
\\{IN}(\\{REF}\\{PARAM}\\{rp2})$: \37\X10:Parse the only parameter\X\\{OUT}%
\\{usage}\\{ESAC}\\{FI}\\{OUT}\\{usage}\\{ESAC};\par
\fi

\M10. The parameter should be one of \.{h}, \.{v} or \.{f}.

\Y\P$\4\X10:Parse the only parameter\X\S$\6
$\\{IF}\\{rp2}\\{IS}\\{no}\\{param}\J\\{THEN}\\{usage}\\{ELIF}\\{prog}\\{OF}%
\\{rp2}=\.{""}\J\\{THEN}\\{usage}\\{ELIF}\\{INT}\\{pos2}\K0$;\5
$\\{char}\in\\{string}((\\{prog}\\{OF}\\{rp2})[1],\39\\{pos2},\39\.{"hvf"})%
\\{THEN}\\{CASE}\\{pos2}\\{IN}\#\|h\#\\{usage},\39\#\|v\#(\\{put}(\\{stand}%
\\{error},\39(\.{"hex:\ version\ "},\39\\{cvs}\\{hex}\\{AFTER}\\{blank}%
\\{AFTER}\\{blank}\\{UPTO}\\{blank},\39\\{newline}));\,\35\\{stop}),\39\#\|f\#%
\X11:Get the file name\X\\{ESAC}\\{ELSE}\\{error}(\.{"unknown\ parameter:\ "}+%
\\{prog}\\{OF}\\{rp2})\\{FI}$\par
\U9.\fi

\M11. The file name may be provided as a string denotation or as just the
required characters.

\Y\P$\4\X11:Get the file name\X\S$\6
\4$\\{IF}\\{UPB}\\{val}\\{OF}\\{rp2}/=1\J\\{THEN}\\{usage}\\{ELSE}\\{CASE}(%
\\{val}\\{OF}\\{rp2})[1]\\{IN}(\\{REF}\\{PARAM}\\{rp3})$: \37$\\{IF}\\{rp3}%
\\{IS}\\{no}\\{param}\J\\{THEN}\\{usage}\\{ELIF}\\{prog}\\{OF}\\{rp3}=\.{""}\J%
\\{THEN}\\{usage}\\{ELSE}\\{infn}\K\\{TC}\\{prog}\\{OF}\\{rp3}\\{FI},\39(%
\\{STRING}\|s)$: \37$\\{IF}\|s=\.{""}\J\\{THEN}\\{usage}\\{ELSE}\\{infn}\K%
\\{TC}\|s\\{FI}\\{OUT}\\{usage}\\{ESAC}\\{FI}$\par
\U10.\fi

\M12. \P$\X1:\X\mathrel{+}\S$\6
\\{STRING}\\{infn};\par
\fi

\M13. !AFTER! and !UPTO! are in the \Web\ general prelude.

\Y\P$\4\X1:\X\mathrel{+}\S$\6
\\{macro}\\{gp}\\{op}\\{upto};\5
\\{macro}\\{gp}\\{op}\\{after}; \\{Reading}\\{the} \&{file} $.\\{We}\\{use}%
\\{the}!\\{stat}!\\{routine}\mathrel{\&{to}}\\{get}\\{the}$ \&{file} $\\{size}.%
\X1:\X=\\{IF}\\{REF}\\{STAT}\\{st}=\\{stat}(\\{infn})$;\5
$\J\\{st}\\{IS}\\{no}\\{stat}\\{THEN}\\{error}(\.{"cannot\ get\ the\ size\ of\
file\ "}+\\{infn})\\{ELSE}\\{sz}\K\\{st}\\{size}\\{OF}\\{st}\\{FI}$;\par
\fi

\M14. \P$\X1:\X\mathrel{+}\S$\6
\\{INT}\\{sz};\par
\fi

\M15. \P$\X1:\X\mathrel{+}\S$\6
\\{macro}\\{no}\\{stat};\5
\\{macro}\\{stat};\5
\\{macro}\\{op}\\{tc};\par
\fi

\M16. Now we can open the file. Because we know its size, overrunning its
end must be regarded as unexpected.

\Y\P$\4\X1:\X\mathrel{+}\S$\6
$\\{IF}\\{open}(\\{inf},\39\\{infn},\39\\{stand}\in\\{channel})/=0\\{THEN}%
\\{error}(\.{"cannot\ open\ file\ "}+\\{infn})\\{ELSE}\\{on}\\{logical}$ %
\&{file}  \6
\&{end} $(\\{inf},\39(\\{REF}\\{FILE}\|f)\\{BOOL}:(\\{error}(\.{"unexpected\
EOF\ on\ "}+\\{idf}(\|f));\,\35\\{SKIP}))\\{FI}$;\par
\fi

\M17. \P$\X1:\X\mathrel{+}\S$\6
\\{FILE}\\{inf};\par
\fi

\M18. We read the file as 16-byte chunks, so we need to round the size down
to the nearest 16 bytes and get the size of any remainder.

\Y\P$\4\X1:\X\mathrel{+}\S$\6
$\\{INT}\\{chunks}=\\{ABS}(\\{BIN}\\{sz}\\{SHR}4),\39\\{remainder}=\\{ABS}(%
\\{BIN}\\{sz}&16\\{rf})$;\par
\fi

\M19. \P$\X1:\X\mathrel{+}\S$\6
$\\{FOR}\|i\\{TO}\\{chunks}\\{DO}[16]\\{CHAR}\\{chunk}$;\5
$\\{get}\\{bin}(\\{inf},\39\\{chunk})$;\5
$\\{print}\\{lines}((\\{BIN}(\|i-1)\\{SHL}4),\39\\{chunk})\\{OD}$;\par
\fi

\M20. There may be an odd-sized chunk at the end.

\Y\P$\4\X1:\X\mathrel{+}\S$\6
$\\{IF}\\{remainder}/=0\\{THEN}[\\{remainder}]\\{CHAR}\\{chk}$;\5
$\\{get}\\{bin}(\\{inf},\39\\{chk})$;\5
$\\{print}\\{lines}(\\{BIN}\\{sz}&\\{NOT}16\\{rf},\39\\{chk})\\{FI}$;\par
\fi

\M21. Now we can close the files.

\Y\P$\4\X1:\X\mathrel{+}\S$\6
$\\{close}(\\{inf})$; $\\{close}(\\{argf})\\{Printing}\\{the}\\{lines}.\\{This}%
\\{routine}\\{must}\\{be}\\{able}\mathrel{\&{to}}\\{deal}$ \6
\&{with} $\|a\\{short}\\{line}.\X1:\X=\\{PROC}\\{print}\\{lines}=(\\{BITS}%
\\{addr},\39[\,]\\{CHAR}\\{data})\\{VOID}:\\{BEGIN}\\{print}((\\{HEX}\\{addr},%
\392\ast\\{blank}))$;\5
\X22:Print the first 8 bytes\X;\6
$\\{IF}\\{UPB}\\{data}>8\\{THEN}\\{print}(\\{blank})$;\5
\X23:Print any more bytes\X\\{FI};\5
\X24:Print the characters on the next line\X\\{END};\par
\fi

\M22. We have to be able to convert a character into two hexadecimal
digits. For this, the operand of !HEX! should be of mode !SHORT SHORT
BITS!. This macro does the conversion.

binhex(c) = HEX SHORTEN SHORTEN BIN ABS c

\Y\P$\4\X22:Print the first 8 bytes\X\S$\6
$\\{FOR}\|i\\{TO}\\{UPB}\\{data}\\{MIN}8\\{DO}\\{print}((\\{binhex}(\\{data}[%
\|i]),\39\\{blank}))\\{OD}$\par
\U21.\fi

\M23. Only the bounds differ for printing any more bytes.

\Y\P$\4\X23:Print any more bytes\X\S$\6
$\\{FOR}\|i\\{FROM}9\\{TO}\\{UPB}\\{data}\\{DO}\\{print}((\\{binhex}(\\{data}[%
\|i]),\39\\{blank}))\\{OD}$\par
\U21.\fi

\M24. Now we can print the characters on the next line, lined up with
their corresponding hexadecimal values.

\Y\P$\4\X24:Print the characters on the next line\X\S$\ $\\{print}((%
\\{newline},\3910\ast\\{blank}))$;\5
\\{FOR}\|i\\{TO}\\{UPB}\\{data}\\{MIN}8\\{DO}\X25:Print a rendered character\X%
\\{OD};\5
$\\{IF}\\{UPB}\\{data}>8\\{THEN}\\{print}(\\{blank})$;\5
\\{FOR}\|i\\{FROM}9\\{TO}\\{UPB}\\{data}\\{DO}\X25:Print a rendered character\X%
\\{OD}\\{FI};\5
$\\{print}(\\{newline})$\par
\U21.\fi

\M25. As clarified in the introduction, each character is either printed
as is if it is in the range !16r20:16r7e!, or as \verb*$\^\|X$ if it
is a control character, or else as a dot.

\Y\P$\4\X25:Print a rendered character\X\S$\6
$\\{print}(\\{IF}\\{CHAR}\\{di}=\\{data}[\|i];\,\35\\{ABS}\\{di}\G\\{ABS}%
\\{blank}&\\{ABS}\\{di}\L\\{ABS}\.{"\~"}\\{THEN}\\{blank}+\\{di}+\\{blank}%
\\{ELIF}\\{ABS}\\{di}<\\{ABS}\\{blank}\\{THEN}\.{"\^"}+\\{REPR}(\\{ABS}%
\\{di}+64)+\\{blank}\\{ELSE}\\{blank}+\.{"."}+\\{blank}\\{FI})\\{System}%
\\{dependencies}.\\{Check}\\{these}$  \6
\&{if} $\\{you}\\{are}\\{porting}\\{this}$ \6
\4\&{program}\1\  \37$\mathrel{\&{to}}\\{another}\\{platform}.$\par

\Us24\ET24.\fi


\inx
\:\\{able}, 21.
\:\\{ABS}, 18, 25.
\:\\{addr}, 21.
\:\\{AFTER}, 10.
\:\\{after}, 13.
\:\\{another}, 25.
\:\\{are}, 25.
\:\\{arg}, 5.
\:\\{argf}, 5, 6, 9, 21.
\:\\{be}, 21.
\:\\{BEGIN}, 21.
\:\\{bin}, 19, 20.
\:\\{BIN}, 18, 19, 20.
\:\\{binhex}, 22, 23.
\:\\{BITS}, 21.
\:\\{blank}, 10, 21, 22, 23, 24, 25.
\:\\{BOOL}, 16.
\:\\{CASE}, 9, 10, 11.
\:\\{channel}, 5, 16.
\:\\{CHAR}, 8, 19, 20, 21, 25.
\:\\{char}, 10.
\:\\{Check}, 25.
\:\\{chk}, 20.
\:\\{chunk}, 19.
\:\\{chunks}, 18, 19.
\:\\{close}, 21.
\:\\{CONTEXT}, 1.
\:\\{cvs}, 10.
\:\\{data}, 21, 22, 23, 24, 25.
\:\\{deal}, 21.
\:\\{dependencies}, 25.
\:\\{di}, 25.
\:\\{DO}, 19, 22, 23, 24.
\:\\{ELIF}, 9, 10, 11, 25.
\:\\{ELSE}, 9, 10, 11, 13, 16, 25.
\:\\{END}, 21.
\:\\{error}, 7, 8, 10, 13, 16.
\:\\{ESAC}, 9, 10, 11.
\:\\{exit}, 8.
\:\\{FI}, 9, 10, 11, 13, 16, 20, 21, 24, 25.
\:\\{FILE}, 6, 16, 17.
\:\\{FINISH}, 3.
\:\\{FOR}, 19, 22, 23, 24.
\:\\{FROM}, 23, 24.
\:\\{get}, 9, 13, 19, 20.
\:\\{gp}, 13.
\:\\{hex}, 1, 10.
\:\\{HEX}, 21.
\:\\{idf}, 16.
\:\\{IF}, 9, 10, 11, 13, 16, 20, 21, 24, 25.
\:\\{IN}, 9, 10, 11.
\:\\{inf}, 16, 17, 19, 20, 21.
\:\\{infn}, 11, 12, 13, 16.
\:\\{INT}, 10, 14, 18.
\:\\{IS}, 9, 10, 11, 13.
\:\\{line}, 21.
\:\\{lines}, 19, 20, 21.
\:\\{logical}, 16.
\:\\{macro}, 13, 15.
\:\\{MIN}, 22, 24.
\:\\{msg}, 8.
\:\\{must}, 21.
\:\\{newline}, 7, 8, 10, 24.
\:\\{no}, 9, 10, 11, 13, 15.
\:\\{NOT}, 20.
\:\\{OD}, 19, 22, 23, 24.
\:\\{OF}, 9, 10, 11, 13.
\:\\{on}, 16.
\:\\{op}, 13, 15.
\:\\{open}, 5, 16.
\:\\{OUT}, 9, 11.
\:\\{param}, 9, 10, 11.
\:\\{PARAM}, 9, 11.
\:\\{params}, 2.
\:\\{platform}, 25.
\:\\{porting}, 25.
\:\\{pos2}, 10.
\:\\{print}, 19, 20, 21, 22, 23, 24, 25.
\:\\{Printing}, 21.
\:\\{PROC}, 7, 8, 21.
\:\\{prog}, 10, 11.
\:\\{PROGRAM}, 1.
\:\\{put}, 7, 8, 10.
\:\\{Reading}, 13.
\:\\{REF}, 9, 11, 13, 16.
\:\\{remainder}, 18, 20.
\:\\{REPR}, 25.
\:\\{rf}, 18, 20.
\:\\{routine}, 13, 21.
\:\\{rp1}, 9.
\:\\{rp2}, 9, 10, 11.
\:\\{rp3}, 11.
\:\\{SHL}, 19.
\:\\{short}, 21.
\:\\{SHR}, 18.
\:\\{size}, 13.
\:\\{SKIP}, 16.
\:\\{st}, 13.
\:\\{stand}, 7, 8, 10, 16.
\:\\{standard}, 2.
\:\\{stat}, 13, 15.
\:\\{STAT}, 13.
\:\\{stop}, 7, 10.
\:\\{STRING}, 11, 12.
\:\\{string}, 10.
\:\\{System}, 25.
\:{system dependencies}, 1, 3.
\:\\{sz}, 13, 14, 18, 20.
\:\\{tc}, 15.
\:\\{TC}, 11.
\:\\{the}, 13, 21.
\:\\{THEN}, 9, 10, 11, 13, 16, 20, 21, 24, 25.
\:\\{these}, 25.
\:\\{this}, 25.
\:\\{This}, 21.
\:\&{to}, \[25].
\:\\{TO}, 19, 22, 23, 24.
\:\\{UPB}, 9, 11, 21, 22, 23, 24.
\:\\{upto}, 13.
\:\\{UPTO}, 10.
\:\\{usage}, 7, 9, 10, 11.
\:\\{use}, 13.
\:\\{USE}, 1.
\:\\{val}, 9, 11.
\:\\{VOID}, 1, 7, 8, 21.
\:\\{We}, 13.
\:\\{you}, 25.
\fin
\:\X11:Get the file name\X
\U10.
\:\X2:Library preludes\X
\U1.
\:\X10:Parse the only parameter\X
\U9.
\:\X25:Print a rendered character\X
\Us24\ET24.
\:\X23:Print any more bytes\X
\U21.
\:\X24:Print the characters on the next line\X
\U21.
\:\X22:Print the first 8 bytes\X
\U21.
\con
